\documentclass[16pt]{article}
% \documentclass[twocolumn]{article}
\usepackage{ctex}
% \usepackage{cite}
\usepackage{geometry}  %页面布局
\usepackage{listings} %代码环境
\usepackage{color}
\usepackage[dvipsnames,svgnames,x11names]{xcolor} %整合12种色彩模式
\usepackage{setspace} %行距
\usepackage{marginnote} %边注
\usepackage{verbatim} %代码环境
\usepackage{paralist} %列表的行间距
\usepackage{amsmath} %公式,ams版本的equation环境可以嵌入次环境
\usepackage{amsthm} %定理,提供proof环境来输出证明
\usepackage{amsfonts} %\mathfrak和\mathbb需要
\usepackage{mathrsfs} %\mathscr{text}需要
\usepackage{graphicx} %插入图片的宏包
\usepackage{subfig} %图片共享有一个子标题
\usepackage{pstricks} %可以直接在文档中插入绘图命令
\usepackage{pspicture}
\usepackage{booktabs} %三线表横线的的粗细
\usepackage{tikz} %前端TikZ调用,底层PGF系统驱动
\usepackage{pst-pdf} %生成包含PSTricks图形的EPS,根据需要转为
\usepackage{multirow} %横跨几行的宏包
\usepackage{warpcol} %调整小数点和数位对齐工作
\usepackage{longtable} %表格太长,用longtable取代tabular
\usepackage{tabularx} %控制整个表格的宽度
\usepackage{rotating} %设置宽表格时使用,用sidewaystable环境替代table
% \usepackage[table]{xcolor}
\usepackage{colortbl} %表格颜色
\usepackage{xltxtra} %XeLaTex标志符号显示使用
\usepackage{texnames} %BibTeX标志使用
\usepackage{mflogo} %METAFONT宏包使用
\usepackage{natbib} %文献
\usepackage{hyperref} %超链接功能
% \usepackage[active,tightpage,xetex]{preview} %如果使用xdvipdfmx,直接生成pdf,文件体积小
%设置纸张布局和页边距
%quad是空格
\geometry{a4paper,left=3.18cm,right=3.18cm,top=2.4cm,bottom=2.4cm}
\graphicspath{{E:/Latex/image}} %事先指定图片的路径
\hypersetup{
    colorlinks=true,
    linkcolor=black
    }
%消除默认生成目录的红框
% \setlength{\parindent}{2em}
%设置段落间的距离
% \addtolength{\parskip}{3pt}

% 设置插入代码的样式
% \lsset和\lstdefinestyle{language-name}性质一样,不过后者的优先级更高

\lstset{
    backgroundcolor = \color{lightgray!30},
    basicstyle      =   \sffamily,      %基本代码风格
    keywordstyle    =   \bfseries,      %关键字风格
    commentstyle    =   \rmfamily\itshape, %注释风格,斜体
    stringstyle     =   \ttfamily,      %字符串风格
    flexiblecolumns,
    numbers         =   left,           %行号的位置在左边
    showspaces      =   false,          %是否显示空格
    numberstyle     =   \zihao{-5}\ttfamily,    %行号的样式,小五号
    %showstringspace =   false,
    captionpos      =   t,              %代码名字所呈现的位置,t指的是top上面
    frame           =   lrtb,           %显示边框
}

% \lstset{
%   backgroundcolor = \color{lightgray!30},
%   keywordstyle    = \color{blue},
%   stringstyle     = \color{brown},
%   basicstyle      = {\small\ttfamily},
%   breaklines      = true,
%   tabsize         = 4,
%   gobble          = 2,
%   numbers         = left,
%   numberstyle     = \tiny\emptyaccsupp,
%   frame           = single,
%   xleftmargin     = \ccwd,
%   numbersep       = \ccwd,
%   columns         = fullflexible,
% %  emphstyle       = {\color{blue}\small\ttfamily},
% %  emph            = {mkdir,rmdir,sudo,mount,umount,rm},
% }

\lstdefinestyle{Python}{
    language        =   Python, % 语言选Python
    basicstyle      =   \zihao{-5}\ttfamily,
    numberstyle     =   \zihao{-5}\ttfamily,
    keywordstyle    =   \color{blue},
    keywordstyle    =   [2] \color{teal},
    stringstyle     =   \color{magenta},
    commentstyle    =   \color{red}\ttfamily,
    breaklines      =   true,   % 自动换行,建议不要写太长的行
    columns         =   fixed,  % 如果不加这一句,字间距就不固定,很丑,必须加
    basewidth       =   0.5em,
}

\lstdefinestyle{LaTeX}{
    language        =   LaTeX, % 语言选LaTeX
    basicstyle      =   \zihao{-5}\ttfamily,
    % numberstyle     =   \zihao{-5}\ttfamily,
    keywordstyle    =   \color{blue},
    keywordstyle    =   [2] \color{teal},
    stringstyle     =   \color{magenta},
    commentstyle    =   \color{red}\ttfamily,
    breaklines      =   true,   % 自动换行,建议不要写太长的行
    columns         =   fixed,  % 如果不加这一句,字间距就不固定,很丑,必须加
    basewidth       =   0.5em,
}


\title{LaTeX文档}
\author{Rouliy }
\date{January 2022}


\begin{document}

\maketitle

%设置目录
\tableofcontents
%设置层次深度
% \setcounter{tocdepth}{2}

\section{入门}
\subsection{关于vs编辑器的一些事}
注意在VS编辑器下编辑时,调出控制台ctrl+shift+Y
\par
全屏模式:ctrl+K 接着按Z; 而后按两次esc可退出
\par
alt+z 自动换行

\LaTeX 空格和\LaTeX{} 空格生成
\subsection{基本语法介绍}
注意用xelatex程序编译源文件生成PDF文件。借助LaTeX工具
\par
\LaTeX{}源文件的语句分为三种:命令(command),数据和注释(comment)。命令以\textbackslash 起始,多为一行;而环境包含一对起始声明和结尾声明,用于多行内容的场合。注释为百分号\%,在编译时忽略。

\subsection{物理与逻辑结构}
\LaTeX{}文档的结构分为物理结构和逻辑结构。其中物理结构:源文件的组织形式,包括序言(preamble)和正文两部分。(现发行的包缺省时引擎一般为pdfTeX)。逻辑结构:最终输出文档的结构,包含标题,目录,章节等。
\par
\begin{lstlisting}
    \documentclass[option]{class} %文档类声明
    \usepackage[option]{package} %引入宏包
\end{lstlisting}

序言:完成一些设置,定义文件呢类型documentclass,引入宏包usepackage,定义命令,环境等;实际内容放在正文部分。\LaTeX{}中常见的文档类型(documentclass):article、report、book,其基本选项如下见表所示。
注:表格在线生成工具网址 https://www.tablesgenerator.com/。
\par
逻辑结构:文档的开头通常有标题,作者,摘要等信息。之后为章节等层次结构,内容分散于层次结构间。具体用法如下:
\par
\begin{lstlisting}
        \title{LaTeX Notes}
        \date{\today}
        \maketitle  %注意\maketitle命令放在最后
        \begin{abstract}
        \end{abstract}%摘要,article和report可以有,book里没
    \end{lstlisting}
\par
\LaTeX{}提供7种层次结构,高级层次可以包含若干低级层次,article中没有chapter,而report和book中支持所有层次
\begin{lstlisting}
        \part{..}           %level -1
        \chapter{..}        %0
        \section{..}        %1
        \subsection{..}     %2
        \subsubsection{}    %3
        \paragraph{}        %4
        \subparagraph{}     %5

        \setcounter{tocdepth}{2}    %设置目录深度
        \tableofcontents            %列出目录

        \chapter*{}          %不想让某些层次的标题出现在目录中,可以加上星号
    \end{lstlisting}


\begin{table}
    \begin{tabular}{ccc}
        \hline
        10pt,11pt,12pt        & 正文字号,第一个为缺省                                               \\
        letterpaper,a4paper   & 纸张尺寸,如[a4paper]                                               \\
        notitlepage,titlepage & 标题后另起新页                                                      \\
        onecolumn,twocolumn   & 栏数                                                                \\
        draft                 & 草稿模式,某些行排得过满,draft模式可以在他们右边标上粗黑线提醒用户 \\
        \hline
    \end{tabular}
\end{table}
\subsection{文字}

\subsubsection{字符输入}
普通字符,控制符,特殊符号,预定义字符串,注音符号等。
\par
特殊的控制符需在字符前加"\textbackslash"表示为:\#,  \$,  \^,  \-,  \_,  \textbackslash。LATEX 中有短划线 (hypen) 、中划线 (en-dash) 和长划线 (em-dash) 。短划线又称连字符,用来连接单词;中划线用来连接数字,可以通过重复两次短划线得到;长划线类似于中文的破折号,重复三次短划线。为了便于比较,这里也给出数学模式的减号。
\begin{lstlisting}
        computer-aided
        1840--2010
        to be---or not to be
        $1-1=0$ 
    \end{lstlisting}
实现的结果:
computer-aided
1840--2010
to be---or not to be
$1-1=0$


\subsubsection{字体样式}
拉丁的文字体:衬线字体(roman,serif),无衬线字体(sansserif),"sans"来源于法语,表示"没有"。还有等宽字体(monospace,typewriter),衬线字体有修饰[类似于中文的宋体,楷体,仿宋等],无衬线字体则类似于黑体。\emph{强调的文字},\underline{画下划线的文字}。
\par

\textrm{roman字体,衬线字体有修饰} \par
\textsf{sans serif,无衬线字体,类似黑体} \par
\textsc{smallcaps,小型大写字母,大写字母的形状同,但尺寸接近小写字母} \par
\textbf{粗体}   \par
\textmd{半粗体} \par
\textit{斜体,修饰精细,对原字体进行了重新设计,版本问题} \par
\textsl{伪斜体,比斜体要宽,版本问题}   \par
\texttt{等宽字体}
% \uwave{字体波浪线} \par
% \sout{strike-out} \par


\subsubsection{换行,换页和断字}
通常用\LaTeX{}来自动换行,用 \textbackslash \textbackslash 或 \textbackslash newline来强制换行,\textbackslash newpage来强制换页。 \par
\LaTeX{}会自动断字(hyphenate),使得每个字分布均匀,有时也需要显示指明断字位置。

\subsubsection{长度}
精确排版时,人们需要控制排版对象的尺寸和位置。其中point是传统印刷采用的单位,而big point是Adobe推出的PostScript时的新单位,注意em为相对单位,比如当前字体为11pt时,lem就是11pt;同时ex和mu也是相对单位。 \par
对文章排版对象的尺寸和位置定义了一系列宏变量,以便在排版重用
\begin{lstlisting}
        \setlength{变量名} %设置变量的值
        \addtolength{变量名} %增加变量、
        \newlength{变量名} %定义变量
    \end{lstlisting}

\subsection{对齐和间距}
\subsubsection{段落和对齐}
\LaTeX{}中段落为两端对齐(fully justified),段落分别居右,中,左( \textbackslash raggedright, \textbackslash centering, \textbackslash raggedleft)功能相同。
\begin{lstlisting}
        \begin{flushleft}
            居左\\常用的对齐方式
            \begin{center}
                居中
                \begin{flushright}
                    居右
                \end{flushright}
            \end{center}
        \end{flushleft}
    \end{lstlisting}

\subsubsection{缩进和段间距}
正文的第一段落缺省不缩进首行,可以用宏包identfirst,其中段落首行缩进的距离可以用 \textbackslash parindent变量控制。

\subsubsection{行间距}
段落中相邻基线间的距离,缺省为单倍行距, \textbackslash linespread{1.3} 为一倍半行距,{1.6}为双倍行距。此命令改变正文的行距,目录,脚注,图表标题等。\par
局部文字的行间距用setpace宏包来实现
\begin{lstlisting}
        \usepackage{setspace}
        \singlespacing %单倍行距
        \onehalfspacing %一倍半行距
        \doublespacing %双倍行距
        \setstretch{1.25} %设置任意行距
        \begin{doublespacing}
            双倍行距
        \end{doublespacing}
    \end{lstlisting}


\subsection{特殊段落}

\subsubsection{摘录}
三种摘录环境:quote,quotation,verse。quote两端都缩进。quotation在quote的基础上增加了首行缩进,verse比quote多了第二行起的缩进。
\begin{quote}
    quote,引文两端缩进
\end{quote}
\begin{quotation}
    在quote的基础上,quotation增加了首行缩进
\end{quotation}
\begin{verse}
    verse,引文两端缩进 \par
    不过是第二行增加了缩进
\end{verse}

\subsubsection{原文打印:源代码和命令}
\textbf{文档中的命令和源代码通常使用等宽字体,即原文打印}。正文中插入少量的可以使用 \textbackslash verb命令;大段的文字可以使用verbatim环境比较方便。以及fancyvrb和listings宏包
\begin{verbatim}
    verbatim不带星号的版本
    printf("LaTeX");
\end{verbatim}
\begin{verbatim*}
    verbatim不带星号的版本,能够显示空格
    printf(" LaTeX*");
\end{verbatim*}

\subsubsection{脚注}
可以使用 \textbackslash footnote \footnote{这是个脚注},对脚注重新定义,脚注是个计数器(counter),有5中显示格式:
\begin{lstlisting}
        阿拉伯数字 \arabic{counter} 1,2,3
        小写英文字母 \alph{counter} a,b,c
        大写英文字母 \Alph{counter} A,B,C
        小写罗马数字 \roman{counter} i,ii,iii
        大写罗马数字 \Roman{counter} I,II,III
    \end{lstlisting}
\par
重定义\textbackslash thecounter的方法改变显示的格式
\begin{lstlisting}
        \renewcommand{\thefootnote}{\roman{footnote}} 
        %注意此命令只改变之后的脚注序号
        %以后出现的计数器,都可以用\thecounter的方法重新定义
    \end{lstlisting}

\subsubsection{边注}
使用\textbackslash marginpar命令。单面排版时,边注缺省排在页面右边空白处;此命令使用了浮动体来生成边注,\texttt{不能在其他浮动体或脚注内嵌套}。而marginnote宏包不涉及浮动。
\marginnote{边注的内容,使用了marginnote宏包,没有用浮动体} %边注内容表示,位置在上面内容的结束处开始。

\subsubsection{注释}
对于大段文字的注释,使用verbatim包的comment环境
\begin{comment}
大段的注释
\end{comment}


\subsection{列表}
\subsubsection{基本列表}
\LaTeX{}三种基本列表环境:无序,有序,描述列表。
\begin{itemize}
    \item 无序列表
    \item 实例
\end{itemize}
\begin{enumerate}
    \item 有序列表
    \item 演示列表
\end{enumerate}
\begin{description}
    \item[列表] 列表可单独使用,可相互嵌套
    \item[C++] 编程语言
    \item[Java] 编程语言
    \item[HTML] 编程语言
\end{description} \par
其他列表:上述列表的行间距较大,为节省空间可以用paralist宏包,它提供了一系列压缩列表和行间列表环境。\\
\textbf{压缩列表:+compact}
\begin{compactitem}
    \item 无序的压缩列表
    \item 实例
    \item HTML
\end{compactitem}
\begin{compactenum}
    \item 有序的压缩列表
    \item HTMLo
\end{compactenum}
\textbf{行间列表:+前缀inpara-}

\begin{inparaitem}
    \item 行间列表
    \item 显示在一行中
\end{inparaitem}
\par
\begin{inparaenum}
    \item 行间有序
    \item 列表
\end{inparaenum}

\subsubsection{定制列表}
改变列表的符号和编号
%\renewcommand{\labelitemi}{-}
%\renewcommand{\theenumi}{\alph{enumi}}
\begin{lstlisting}
        \renewcommand{command}{def}
    \end{lstlisting}
改变符号后的表格
\begin{itemize}
    \item 无序列表
    \item HTML
\end{itemize}
\begin{enumerate}
    \item 有序列表
    \item HTML
\end{enumerate}

\subsection{盒子}
\subsubsection{初级盒子}
相似与HTML和CSS中的模型。最简单的盒子命令为:\textbackslash mbox和\textbackslash fbox。前者把对象组合,后者则价格外框。
\mbox{010 6278 5001}
\fbox{010 6278 5001}
\subsubsection{中级盒子}
稍微复杂的盒子为,\textbackslash makebox和\textbackslash framebox命令提供宽度和对齐方式选项。居中(默认),居左,右,两端对齐[c,l,r,s]。\par
\makebox[100pt][c]{宽度,对齐方式,内容} \par
\framebox[200pt][s]{两端对齐}
\subsubsection{高级盒子}
大一些的对象,比如整个段落可用\textbackslash parbox命令或minpage环境,两者类似。 \par
语法格式:[外部对齐](t,c,b居顶,中,底对齐)[高度][内部对齐]{宽度}{内容} \par
\fbox{
    \begin{minipage}[c][36pt][t]{300pt}
        高级盒子试验模型,与周围对象的纵向关系为外部对齐。
    \end{minipage}
}
\par
\fbox{
    \parbox[l][36pt][t]{170pt}{
        盒子的内容,fbox带边框
    }
}
\begin{lstlisting}
                \fbox{
            \parbox[c][36pt][t]{170pt}{
                盒子的内容
            }
        }
    \end{lstlisting}

\subsubsection{交叉引用}
(cross reference),对前面的最近一个对象的编号或页码引用。
\begin{lstlisting}
        一个标签 \label{marker}
        第\pageref{marker}页\ref{marker}节
        其中,marker为标签名
    \end{lstlisting}
标签 \label{bq}
第\pageref{bq}页\ref{bq}节,
注意标签第一次编译后会提示警告信息,第二次才有正确的结果。


\section{字体}
字体的概念可以划分为三个层次:\par
\begin{compactenum}
    \item 编码层,字符(包括字母,数字,控制码等)的索引和编码,即字符集(character)和字符编码(character encoding)
    \item 格式层,字型(glyph)的定义描述方法,字体的文件储存格式
    \item 显示层,字体的外在表现形式,如字体的样式或具体的字体
\end{compactenum}
\subsection{字符集和编码的故事}
微软将IBM的代码页称为OEM代码页,自己定义的称为ANSI代码页,比如936(GBK 简体中文),950(Big5 繁体中文)。1981年,中国大陆推出第一个自己的字符集标准GB2312,94×94的表,包括7445个字符。GB2312通常采用双字节EUC-CN编码,所以后者又称为GB2312编码。GB2312在1993年被拓展为GBK,包含21886个字符。GBK不是正式的标准,2000年发布的GB18030包含70244个字符,采用四字节的编码。\par
1990年ISO推出了通用字符集(universal character set,UCS),即ISO10646。UCS有两种编码:双字节的UCS-2和四字节的UCS-4。除了ISO外,统一码联盟(The Unicode Consortium),它于1991年推出了Unicode 1.0,其后双方合并,Unicode 2.0版开始采用与ISO 10646-1相同的编码。 \par
Unicode主要有三种编码:UTF-8,UTF-16,UTF-32。UTF-8使用1-4个8位编码,UTF-16使用1-2个16位编码,和ASCII不兼容。UTF-32用1个32位编码。IETF要求所有的网络协议都支持UTF-8,IMC也建议所有的电子邮件软件都支持UTF-8。
\subsection{字体格式}
\subsubsection{点阵和矢量字体}
电脑字体的数据格式可分为3大类:点阵(bitmap)字体,轮廓(outline)字体[矢量字体]和笔画(stroke-based)字体。 \par
点阵字体通过点阵描述字体;矢量字体通过一组直线段和曲线描述字形,轮廓字体的缺陷在于其所采用的贝塞尔曲线在光栅设备(如显示器和打印机)不能渲染,故需要额外的补偿处理;笔画字体其实也是轮廓字体,不过其描述的不是完整字形,而是笔画,多用于东亚文字。
\subsubsection{常见的字体格式}
轮廓字体格式:Type1,TrueType,OpenType。Type1采用三次的贝塞尔曲线,TrueType采用二次贝塞尔曲线,处理比三次曲线快,但需要更多的点描述。OpenType被认为是Type1和TrueType的超集,既可以使用二次曲线也可以使用三次曲线。
\subsubsection{常见字体}
TEX的缺省字体是Knuth用METAFONT生成的Computer Modern;XETEX的缺省字体是1997年AMS发布的Latin Modern,基于Computer Modern,但是扩展了字符集,其封装格式有Type1和OpenType。

\subsection{字体的应用}
LaTeX,pdflatex,xelatex编译程序,dvips和dvipdfmx驱动,DVI浏览器等分别采用不同的字体技术路线。
\subsubsection{早期技术}
\textbf{latex和DVI} \par
用latex生成DVI时只需要TFM文件,DVI并不包含字形信息,而只包含对字体的引用。DVI浏览器显示DVI时一般使用PK,找不到则调用METAFONT在后台生成。
\par
\textbf{dvips} \par
缺省时,dvips也会查找.pk,没有则自动生成。

\subsubsection{XeTeX}
XETEX用一个XML文件记录系统字体路径,MikTeX用的是localfonts.conf,TeXlive用的是fonts.conf。设置字体时需要字体的引用名,和字体的文件名是不同的概念。\par
XETEX提供的字体命令比较原始,繁琐,而fontspec宏包提供了较好的封装。

\section{数学模式}
在序言部分加载amsmath宏包,提供数学的排版功能。数学模式有两种形式:行间(inline)模式(在正文中插入数学内容)和独立(display)模式(独立排列,可以有编号或没有编号)。 \par
    \begin{lstlisting}
        行间公式用$...$
        无编号的独立公式用\[...\]
        建议不用$$...$$,会和AMS-Latex有冲突
    \end{lstlisting}

    \[ \boxed{E=mc^2} \]
    \begin{equation}
        E=mc^2
    \end{equation}
    \begin{math}
        \textrm{段内的文本你,此处为math:}\sum\limits_{i=1}^n
    \end{math}
    \begin{displaymath}
        \textrm{公式中的文字用textrm命令,段外的文本,此处为displaymath:}\sum\limits_{i=1}^n
    \end{displaymath}

\subsection{基本元素}
\subsubsection{希腊字母}
希腊字母需要用命令输入,大写的希腊字母的命令首字母也是大写。
$\alpha \Gamma \Delta $
\[\epsilon \zeta\ \Theta\]

\subsubsection{上下标和根号}
    \begin{equation}
        \theta = \sigma + \rho
    \end{equation}
    \[x_{ij}^2\quad \sqrt{x}\quad \sqrt[3]{x} \]
\subsubsection{分数}
    用\textbackslash frac命令表示,可根据环境自动调整字号,如行间公式中中小一点,独立公式中大一点。也可以设置分数字号,如\textbackslash dfrac把分数的字号设置为独立公式中的大小,\textbackslash tfrac则将其设置为行间公式的大小。\par
    $\frac{1}{2}  \dfrac{1}{2}$
    \[\frac{1}{2}
    \tfrac{1}{2}\]
\subsubsection{运算符}
小运算符可直接输入如+ - * / = 等可以直接输入。
\[\pm\ \times\ \div\ \cdot\ \cap\ \cup\ \geq\ \leq\ \neq\ \approx\ \equiv \]
\par
和,积,极限,积分等大运算符用\textbackslash sum,\textbackslash prod,\textbackslash lim,\textbackslash int等命令。

\begin{lstlisting}
    quad是空格
    $\sum_{i=1}^n i
    \prod_{i=1}^n
    \lim_{x\to0}x^2
    \int_a^b x^2 dx $
    $\sum\limits_{i=1}^n i
    \prod\limits_{i=1}^n j
    \[\sum\nolimts_{i=1}^n i
    \prod\nolimts_{i=1}^n \]$
\end{lstlisting}
说明:quad是空格 \par
$\sum_{i=1}^n i \quad
\prod_{i=1}^n   \quad
\lim\limits_{x\to0}x^2 \quad
\int_a^b x^2 dx $   
$\sum\limits_{i=1}^n i  \quad
\prod\limits_{i=1}^n j$
\[\sum\limits_{i=1}^n i \quad \prod\limits_{i=1}^n \quad \lim\limits_{x\to0^-}x^2 \quad \int_a^b x^2 dx\]
\par
多重积分表示:
    \begin{lstlisting}
        \iint 二重积分
        \iiint 三重积分
    \end{lstlisting}
    \[\int_{-\infty}^{+\infty}x^3dx\]
    \[\lim_{\lambda \to 0}\sum\limits_{i=1}^n f(\xi _i,\eta _i)\Delta \sigma _i=\iint\limits_{D_{xy}} f(x,y)d\sigma \]
    \[\iiint\limits_{\Omega }x^3 dxdydz\]
\textbf{箭头:}
    $ \rightarrow \Longrightarrow $\textbackslash xleftarrow和\textbackslash xrightarrow。可扩展的箭头:\par
    \[\xleftarrow[x<y]{x+y+z}\]

\subsubsection{注音和标注}
数学注音符号(accent),列出一些长的标注符号。\par
\[\bar{x} \quad \vec{x} \quad \hat{x} \quad \check{x} \quad \mathring{x} \quad \tilde{x} \quad \acute{x} \quad \grave{x} \quad \breve{x}\] \par
速度,加速度:\par
\[\dot{x} \quad \ddot{x} \quad \dddot{x}\]
\subsubsection{分隔符}
    各种括号(),[],\{\},尖括号$\langle \rangle$。 \par
    $\overline{xxx} \quad \overleftarrow{xxx} \quad \underline{xxx} \quad \overleftrightarrow{xxx} \quad \overbrace{xxx} \quad \widehat{xxx} \quad \widetilde{xxx}$
    $\textrm{amsmath宏包推荐公式中用:}\lvert \quad \rvert \quad \lVert \quad \rVert$代替\LaTeX{}中的|。 \par
    同时可以在分隔符前面加\textbackslash big \textbackslash Big \textbackslash bigg \textbackslash Bigg等调整其大小。\par
    \big(,\bigg(,\Big(,\Bigg(。

\subsubsection{省略号}
    省略号用\textbackslash dots \textbackslash cdots \textbackslash vdots \textbackslash ddots等命令表示。
    \begin{lstlisting}
        \dots 下标省略号
        \cdots 中部的省略号
        \vdots 竖向的省略号
        \ddots 斜向的省略号 
    \end{lstlisting}
    \[x_1,x_2,\dots,x_n\quad 1,2,\cdots,n \quad \vdots \quad \ddots\]
\subsubsection{空白间距}
    \textbackslash ,\quad $3/18em; \quad |\,| \quad$ \par
    \textbackslash :\quad $4/18em; \quad |\:| \quad$ \par
    \textbackslash ;\quad $5/18em; \quad |\;| \quad$ \par
    \textbackslash quad\quad $1em; \quad |\quad| \quad$ \par
    \textbackslash qquad\quad $2em; \quad |\qquad| \quad$ \par
    \textbackslash !\quad $-3/18em; \quad |\!| \quad$
\subsubsection{矩阵}
    数学模式下可用array环境生成矩阵。外部对齐方式(t,c,b),列对齐方式(l,c,r)。\textbackslash \textbackslash 用来分隔行和\& 用来分隔列。
    \begin{lstlisting}
        \begin{array}[外部对齐]{列对齐}
            行列内容
        \end{array}
    \end{lstlisting}
    \[
        \begin{array}{ccc}
            x_1     &       x_2     &   \dots \\
            x_3     &       x_4     &   \dots \\
            \vdots  &       \vdots  &   \ddots 
        \end{array}
        \]
    \par
    amsmath的pmatrix,Bmatrix,vmatrix,Vmatrix等环境可以在矩阵两边家伙加上各种分隔符,但它们没有对齐参数。 \par
    \textbackslash smallmatrix命令可以生成行间矩阵。
    A little martix 
    $(
        \begin{smallmatrix}
            a   &   b   \\
            c   &   d
        \end{smallmatrix}
    )$

\subsection{多行公式}
    对应公式太长导致一行放不下,用amsmath提供一些多行公式。
\subsubsection{长公式}
    无需对齐的长公式使用multline环境。需要对齐的长公式使用split环境,但其本身不能单独使用,必须包含在其他的数学环境中,因此也称为次环境,用\textbackslash \textbackslash 和 \& 来分行和设置对齐的位置。
    \begin{multline}
        x = a+b+c+{} \\
        d+e+f+g
    \end{multline}
    \[
        \begin{split}
            x = {} & a+b+c+{} \\
                   & d+e+f+g
        \end{split}
        \]
    \subsubsection{公式组}
    不需对齐的公式组:gather环境。需对齐的公式组用align环境。
    \begin{gather}
        a = b+c+d \\
        x = y+z
    \end{gather}
    \begin{align}
        a  &= b+c+d \\
        x  &= y+z
    \end{align} 
    multline,gather,align等环境都带有*的版本,不生成公式编号。
\subsubsection{分支公式-分段函数的宝典}
    通常用case次环境生成分支公式。 \par
    \[
        y=
        \begin{cases}
            -x, \quad x\leq 0 \\
            x, \quad x>0
        \end{cases}
    \]
\subsection{定理和证明}
    \textbackslash newtheorem命令可用来定义定理之类的环境,其语法如下:\par
    {环境名}[编号延续]{显示名}[编号层次] \par
    \textbackslash newtheorem{环境名envname}[caption]{within}
    以下为定制的四个环境:定义,定理,引理和推论,在一个section内统一编号,引理和推论会延续定理的编号。 \par
    \begin{lstlisting}
        \newtheorem{definition}{定义}[section]
        \newtheorem{theorem}{定理}[section]
        \newtheorem{lemma}[theorem]{引理}
        \newtheorem{corollary}[theorem]{推论}     
    \end{lstlisting} \par
    \newtheorem{definition}{定义}[section]
    \newtheorem{theorem}{定理}[section]
    \newtheorem{lemma}[theorem]{引理}
    \newtheorem{corollary}[theorem]{推论}
    
    \begin{definition}
        引入复数概念:$i^2=-1$
    \end{definition}
    \begin{theorem}
        两点之间线段最短 
    \end{theorem}
    \begin{lemma}
        从A点到B点走直线最快
    \end{lemma}
    \begin{corollary}
        从某一点到另一点的最短路径
    \end{corollary}
    amsthm宏包提供的proof环境可用来输出证明,并且在证明的结尾加上QED符号。演示如下所示: \par
    \begin{proof}[证明正定矩阵可逆]
        A*B=C
    \end{proof}

\subsection{数学字体}
    和文本模式类似,数学模式下也可选用不同的字体样式。\textbackslash mathbb和\textbackslash mathfrak需要amsfonts宏包,\textbackslash mathscr需要mathrsfs,汉字不会显示出来,仅限于在数学模式下选用。
    \[
        ABCXYZ \quad
        \mathrm{ABCXYZ} \quad
        \mathsf{ABCXYZ} \quad
        \mathtt{ABCXYZ} \quad
        \mathcal{ABCXYZ} \quad
        \mathbf{ABCXYZ} \quad
    \]
    \[
        \mathit{ABCXYZ} \quad
        \mathbb{ABCXYZ} \quad
        \mathfrak{ABCXYZ} \quad
        \mathscr{ABCXYZ} \quad
    \]

\section{图形}
    当年Knuth开发Tex时,GIF,JPEG,PNG,EPS等格式还未问世,但通过\textbackslash special命令,让后面的驱动自行决定怎样处理图形。
\subsection{图形概述}
\subsubsection{图形格式}
\LaTeX{}支持点阵格式JPEG和PNG,也支持矢量格式EPS和PDF;对于示意图,首选矢量格式;包含大量自然色彩的图像(如照片)首选JPEG;人工点阵图像应该选择PNG。\par
人们考虑把PostScript作为文档中嵌入图形的标准格式,担心嵌入文档的PostSript会搞破坏,就产生了Encapsulated PostScript(EPS)。同样,担心嵌入HTML的ActiveX,Java Applet,JavaScript中混入恶意代码,故会对它们产生限制,因此EPS就成了\LaTeX{}的标准图格式。
\subsubsection{Driver}
\textbf{pdflatex} \par
    pdflatex支持JPEG,PNG和PDF,但不支持EPS。\LaTeX{}有两个宏包 epstopdf和pst-pdf 可以实时地(on the fly) 把EPS转换为PDF。然而前者有安全漏洞,后者用法繁琐,用户最好还是用其他软件事先把EPS转为PDF。pdflatex包含编译和驱动两种功能
    \par
\textbf{dvipdfm(x)} \par
    dvipdfm支持JPEG,PNG和PDF,不支持EPS,但可以实时调用Ghostscript把EPS转为PDF,同时增加了BMP的支持。\par
    
\textbf{xdvipdfmx} \par
    XeLatex的缺省驱动xdvipdfmx支持BMP,JPEG,PNG,EPS和PDF。xdvipdfmx比dvips,pdflatex和dvipdfmx都好。 \par
\subsubsection{图形优化}
    矢量图形的一个优点是可以无限缩放的,而输出质量不变,尺寸对矢量图形的意义不大。描述矢量图形所需的数据较少,其文件体积也一般较小。 \par
\textbf{图形尺寸} \par
    点阵图像的像素是一种相对尺寸,实际尺寸=像素/分辨率(resolution),最常用的分辨单位是像素/英寸(pixel per inch,PPI),点/英寸(dots per inch,DPI)混用。人们倾向于认为PPI是图形的分辨单位,DPI是硬件设备的输出分辨率。\par
    当图形分辨率和输出分辨率不一致时,会有一个重新采样(resampling)的过程;从高分辨率到低分辨率叫下采样(downsampling),反之叫上采样。高分辨率图形遇到低分辨率设备会形成浪费,低分辨率图形遇到高分辨率设备,看插值的效果。 \par
    点阵图形的分辨率选择:屏幕上阅读(72 PPI),考虑到放大150 PPI,高质量的打印300PPI足够。考虑在\LaTeX{}文档中插入图形,通栏的话宽度是4.8-5.4in(缺省宽度取决于字体的大小),仅用于屏幕阅读的话,宽度400px足够;放大阅读或输出打印需要800px或1600px足够。 \par
    点阵图形尺寸相关的基本操作有:裁剪(crop),改尺寸(resize),改分辨率。\par
    1.裁剪时像素会变少,分辨率不变,缺省输出去尺寸也就变小。\par
    2.改变尺寸时像素改变,分辨率不变,缺省输出去尺寸改变。\par
    3.单纯改变图形文件分辨率时,像素不变,缺省尺寸相应改变 \par
    综上,点阵图形的信息量取决于像素,图形的分辨率为“建议”缺省输出尺寸,不影响图形质量。\par
    \textbf{色彩深度} \par
    色彩深度(color depth)是每个像素所用颜色的位数。1位表示2色;2位表示4色,最早用于CGA显卡;16位表示65536色,又称为高彩;24位表示真彩;30-48位表示为深彩,一般照片用24位足够,常用是将图形的色深从高改低。\par
    \subsubsection{图形转换和处理}
    命令行界面的图像处理软件ImageMagick,图形界面用Paint.NET。\par
    \textbf{ImageMagick} \par
    ImageMagick包含多个命令行程序,最常见的是convert。比如下面的命令把BMP转换为PNG: \par
    \begin{lstlisting}
        convert fig.bmp fig.png
    \end{lstlisting} \par
    因为Windows中有个区分格式的同名程序,故在Windows下使用ImageMagical则需要全路径,或在PATH环境变量中把ImageMagick的路径放到sysytem32前面。 \par
    \begin{lstlisting}
        convert fig.bmp -crop 300×200+10+10 fig.jpg
        %裁剪并转换格式,截取从(10,10)开始300×200的像素
        convert fig.jpg -crop 300×200+10+10 -resize 30×20 fig1.jpg
        %裁剪并缩放
        convert fig.jpg -resize !300×200 fig1.jpg
        %强制缩放不考虑长宽比
        convert fig.jpg -density 300 fig1.jpg
        %分辨率改为300PPI
        convert fig.jpg -resample 300 fig1.jpg
        %分辨率改为300PPI,像素增加,缺省输出尺寸维持不变
    \end{lstlisting} \par
    ImageMagick功能强大,参数选项很多,但缩小图像做缩略图时不是很清晰 \par
    \textbf{转换为EPS} \par
    ImageMagick 转换 EPS 的方法如下。如果是 BMP 文件,最好先压缩JPEG 或 PNG,再转为 EPS,这样生成的 EPS 会比较小。\par
    \begin{lstlisting}
        convert fig.png eps3:fig.eps
    \end{lstlisting} \par
    另一种方法是用虚拟打印机生成 EPS,它的优点是可以把几乎所有文件“打印”成 EPS。推荐 Bullzip PDF Printer,它可以把各种文件打印成PS、 EPS、 PDF、 BMP、 JPEG、 PCX、 PNG、 TIFF 等格式。

\subsection{插入图形}
\subsubsection{范围框}
    latex编译程序处理图形文件时需要范围框(bounding box)。pdflatex和xelatex出现的较晚。EPS是一种嵌入图形格式,其范围框如下:\par
    \begin{lstlisting}
        %!PS-Adobe-3.0 EPSF-3.0
        %%BoundingBox: 5 5 105 105
    \end{lstlisting} \par
    前两个参数是图形左上角的坐标(一般为原点),后两个参数是右下角的坐标,缺省的长度单位是bp。 \par
    latex在编译源文件时通过范围框为插图预留空间;它输出的DVI只记录图形尺寸和文件名,具体的图形处理由后面驱动负责,无范围框时,latex则会报错。 \par
    两种方法为点阵图形提供范围框:1.准备一个单独的范围框文件;2.在插入图形时加范围框参数,用latex推荐用第二种方法。dvipdfm附带的ebb程序可以检查JPEG和PNG,生成范围框文件,bp值 = 像素/分辨率*72。 \par
\subsubsection{基本命令}
    面向\LaTeX{}2的graphics和graphicx宏包,后者基于前者,语法更简单,功能更强大。插图命令的基本语法如下:\par
    \begin{lstlisting}
        \usepackage[dvipdfm]{graphicx}
        \includegraphics[bb=0 0 300 200]{fig.png}
    \end{lstlisting} \par
    引用宏包时可加驱动选项,使用latex缺省为dvips,dvipdfm(x)用dvipdfm;pdflatex和xelatex分别用pdftex和xetex(可以不加驱动)。使用latex时,没有生成.bb文件,需要或加范围框参数。
\subsubsection{图形操作}
    \textbackslash includegraphics命令有一些参数选项用于缩放,裁剪等图形操作:\par
    1.width=x,height=y; \quad scale=s; \par 
    2.keepaspectratio \quad  保持图形比例,确保高,宽比例不会失调。 \par
    3.angle=a \quad 逆时针旋转的角度,单位度 \par
    4.origin=h[l,c,r]v[t,c,b] \quad 旋转中心,缺省在左下。水平方向可选左,中,右(l,c,r),垂直方向选上,中,下(t,c,b)。 \par
    5.totalheight=h \quad 总高度,最高,最低两点间垂直距离。 \par
    6.viewport=x1,y1,x2,y2 \quad 可视区域左上角和右下角坐标,缺省单位bp。 \par
    7.trim=l,b,r,t \quad 左,下,右,上四边的裁剪值,缺省单位bp。\par
    8.clip \quad 是否真正裁剪,配合viewport或trim使用。 \par
    9.page=n \quad 选页,用于多页图形文件。 \par
\subsubsection{文件名和路径}
    \begin{lstlisting}
        includegraphics[width=60pt]{homer.pdf} 原图
        includegraphics[width=80pt]{homer.pdf} 放大
        includegraphics[width=80pt,height=100pt]{homer.pdf} 变形
        includegraphics[width=80pt,height=100pt,keepaspectratio]{homer.pdf} 
        保持比例不变形
    \end{lstlisting}
    \begin{lstlisting}
        图形旋转
        \includegraphics[angle=90]{homer.pdf}
        \includegraphics[angle=90,origin=c]{homer.pdf}
    \end{lstlisting}
    \begin{lstlisting}
        \DeclareGraphicsExtension{.eps,.mps,.pdf,.png}
        指定后缀列表让编译程序自行查找
        \DeclareGraphicsRule{*}{eps}{*}{}
        指出未知后缀的都是EPS
        \graphicspath{{c:/file/iamge}} 绝对路径
        \graphicspath{{./img/}} 相对路径
        \graphicspath{{}} 多个路径
        综上,前三个为设置的缺省搜索路径,注意文件名和路径名不能有空格
        路径名用正斜杠/,路径名用/结尾。
    \end{lstlisting}
\subsubsection{figure环境}
    插图通常需要占据大块的空间,因此在文字处理软件中经常要调整插图的位置。而figure环境则可以自动完成这样的任务,这种自动调整位置的环境称为浮动环境(float)。 \par
    \begin{lstlisting}
        h(here)t(top)b(bottom)p(page)
        这里,页顶,页尾,浮动页
        四个字母都写上表示放哪里都无所谓
        \begin{figure}[htbp]
            \centering
            \includegraphics{1.jpg}
            \caption{图像标题}
            \label{fig:1.jpg}
            \label一般放在标题之后
        \end{figure}
    \end{lstlisting}

\subsubsection{插入多幅图形}
    \textbf{并排摆放,共享标题} \par
    当我们需要两幅图形并排时,共享标题,可以在figure环境中使用两个\textbackslash includegraphics命令。 \par
    \begin{lstlisting}
        \begin{figure}
            \centering
            \includegraphics{left.pdf}
            \includegraphics{right.pdf}
            \caption{并排排放,共享标题}
        \end{figure}
    \end{lstlisting}
    \textbf{并排摆放,各有标题} \par
    各有标题,可以使用minipage环境,caption命令会把环绕它的minipage环境变成figure环境。\par
    \begin{lstlisting}
        \begin{figure}[htbp]
            \centering 使得minipage整体居中
            \begin{minipage}{60pt}
                \centering 使内容整体居中
                    \includegraphics{left.pdf}
                    \caption{小标题1}
            \end{minipage}
            \hspace{10pt}
            \begin{minipage}{60pt}
                \centering 使内容整体居中
                \includegraphics{right.pdf}
                \caption{小标题2}
            \end{minipage}
        \end{figure}
    \end{lstlisting} \par
    \begin{figure}[htbp]
        \centering %使得minipage整体居中
        \begin{minipage}{200pt}
            \centering %使内容整体居中
                \includegraphics{2.jpg}
                \caption{小标题1}
        \end{minipage}
        \hspace{10pt}
        \begin{minipage}{200pt}
            \centering %使内容整体居中
            \includegraphics{3.jpg}
            \caption{小标题2}
        \end{minipage}
    \end{figure}

\textbf{共享标题,各有子标题} \par
    使用subfig宏包,其提供\textbackslash subfloat命令。
    \begin{figure}[htbp]
        \centering
        \includegraphics{image/1.jpg}
        \caption{插入的图片}
    \end{figure}

        \begin{lstlisting}
        \begin{figure}
            \centering
            \subfloat[左边子标题]{
                \label{fig:a} %避免子标题折行,嵌套minipage
                \begin{minipage}[t]{60pt}
                    \centering
                    \includegraphics{2.jpg}    
                \end{minipage}
            }
        \hspace{10pt}%
            \subfloat[右边子标题]{
                \label{fig:b}
                \begin{minipage}[t]{60pt}
                    \centering
                    \includegraphics{3.jpg}    
                \end{minipage}
            }
        \caption{共享的标题}
        \label{fig:subfig} %标签在引用或交叉引用的时候使用
        \end{figure}   
    \end{lstlisting}

    \begin{figure}[htbp]
        \centering
        \subfloat[左边子标题]{
            \label{fig:a}
            \includegraphics{2.jpg}
        }
    \hspace{10pt}%
        \subfloat[右边子标题]{
            \label{fig:b}
            \includegraphics{3.jpg}
        }
    \caption{共享的标题}
    \label{fig:subfig1}
    \end{figure} 

    \begin{figure}
        \centering
        \subfloat[左边子标题]{
            \label{fig:c} %避免子标题折行,嵌套minipage
            \begin{minipage}[t]{200pt}
                \centering
                \includegraphics{2.jpg}    
            \end{minipage}
        }
    \hspace{10pt}%
        \subfloat[右边子标题]{
            \label{fig:d}
            \begin{minipage}[t]{200pt}
                \centering
                \includegraphics{3.jpg}    
            \end{minipage}
        }
    \caption{共享的标题}
    \label{fig:subfig2} %标签在引用或交叉引用的时候使用
    \end{figure}   

\subsection{矢量绘图}
   色彩模型分类:1.用于照相,摄影,电视,电脑的三原色模型;2.用于彩色印刷的CMYK四色模型。前者是加色模型,后者是减色模型。\par
   24位真彩模型中,三原色(红,绿,蓝)各用8位表示,取值从0到255。HTML常用16进制表示。RGB模型用的为笛卡尔坐标,色彩变化不够连续。若将其坐标系改为圆柱坐标系,就是HSL和HSV模型。\par
   CMYK(蓝,青,洋红,黄)模型默认纸张原本为白色或浅色背景,在上面印刷某种颜料就会减少颜色光的反射,看上去是它的补色。印青色就会得到红色,印洋红得到绿,印黄得到蓝,青、洋红、黄三种颜色都印上就会得到黑色。彩色印刷有分色和套色过程,如果图形上有黑色直接拿黑颜料印刷会减少成本。 CMYK 中的字母K就代表key black。 \par
   \textbf{预定义和自定义颜色} \par
   xcolor宏包预定义颜色:19种基本颜色,68种dvips颜色,151种svg颜色,317种Unix/X11颜色。后3类颜色使用时需要加相应的预定义颜色集合选项:\par
   \begin{lstlisting}
       \usepackage[dvipsnames]{xcolor}
       \usepackage[svgnames]{xcolor}
       \usepackage[x11names]{xcolor}
   \end{lstlisting} \par
   也可以使用\textbackslash definecolor命令来自定义更多的颜色: \par
   \begin{lstlisting}
    语法:\definecolor{名称}{模式}{参数} 
    \definecolor{d_red}{RGB}{255,0,0} 
    \definecolor{d_green}{HTML}{00FF00} 
   \end{lstlisting} \par
   \textbf{彩色文字} \par
   设置文字颜色可以使用\textbackslash textcolor命令。 \par
   语法: \textbackslash textcolor{名称}|[模式]{代码}{文字} \par
   \textcolor[RGB]{1,1,200}{颜色} \quad \textcolor{red}{红} \quad \textcolor{pink}{粉红} \quad \textcolor{gray}{灰色字体} \par
   \textbf{彩色盒子} \par
   \textbackslash colorbox命令可以生成有彩色背景的盒子,与\textbackslash textcolor类似。\textbackslash fcolorbox命令又给彩色盒子加上边框。 \par
   \begin{lstlisting}
       \colorbox{SkyBlue}{}
       \fcolorbox{Silver}{SkyBlue}{}
       \fcolorbox{RoyalBlue}{Lavender}{}
   \end{lstlisting}
   \colorbox{SkyBlue}{}
   \fcolorbox{Silver}{SkyBlue}{}
   \fcolorbox{RoyalBlue}{Lavender}{}
   \begin{figure}[htbp]
        \begin{minipage}{500pt}
            \centering
            \includegraphics{6.jpg}
        \end{minipage}
       \caption{颜色表}
   \end{figure}

\subsubsection{绘图工具概述}
   与\LaTeX{}配套使用的矢量绘图工具主要有3种:META-POST,PSTricks,PGF。概述如下: \par
   \begin{itemize}
       \item 工作方式:METAPOST离线绘图,生成的MPS(一种特殊的EPS);PSTricks和PGF都采用在线绘图的方式,即在\LaTeX{}文档内直接使用绘图命令。 
       \item 兼容性:METAPOST生成的MPS需要先转换为PDF才能被pdflatex使用;PSTricks生成的EPS和pdflatex不兼容;PGF提供针对各种驱动的接口,兼容性最好。
       \item 功能:PSTricks有PostScript作后盾,功能最强;METAPOST擅长处理数学内容;PGF擅长流程图。
   \end{itemize} \par
   也可以考虑一些面向\LaTeX{}的绘图前端,如Dia和Ipe,或一些更通用的软件,比如gnuplot和Inkscape。

\section{Metapost}
   METAFONT只支持黑白,是用来设计字体的。 METAPOST输出的是EPS,而且支持彩色,同时可以在图形上加文字标注,甚至插入TEX源码。 \par
   \subsection{准备工作}
    METAPOST的缺省长度单位是bp,也可以使用其他单位。
    \begin{lstlisting}
        u := 10pt %变量赋值符号为:=
        beginfig(1); %图形起始
        ...     %绘图命令
        endfig;
        beginfig(2);
        ...
        endfig;
        end
    \end{lstlisting}\par
    变量赋值符号为:=,且一个变量在同一源文件中只需定义一次。一个METAPOST源文件(.mp)。代码中每行语句以分号结尾,注释以百分号起始,回头命令包含在一对起始和结尾声明之间,文件结尾也有一个即为声明。 \par
    我们可以用命令行程序mpost编译源文件,生成MPS(特殊的EPS),然后将MPS插入\LaTeX{}中使用。\par
    METAPOST提供一个变量来设置输出文件名,将下面的代码加到源文件的头部,编译输出的文件名就会是“fig-01.mps,fig-02.mps” \par
    \begin{lstlisting}
        outputtemplate "%j-%2c.mps";  %加在源文件的头部
        outputtemplate "flowchart.mps" %加在每个图形前面
    \end{lstlisting}
    xelatex不能识别MPS格式,故需要使用.eps后缀,或使用以下命令: \par
    \begin{lstlisting}
        \DeclareGraphicsRule{.mps}{eps}{.mps}{}
    \end{lstlisting}
    \begin{figure}
        \centering
        \includegraphics{8.jpg}
    \end{figure}

\subsection{基本图形对象}
\subsubsection{点和线} METAPOST的缺省直径为0.5bp,可以用withpen选项为\textbf{单个绘图命令}设置画笔,也可以用pickup命令为\textbf{之后所有的绘图命令}设置画笔。用draw命令将几个点以直线连接起来,drawdot命令在指定坐标画点。\par
    filenametemplate "line.eps";
    u := 10pt ;%变量赋值符号为:=
    beginfig(1);
    draw (0,0)--(4u,0)--(2u,2u)--(0,0) withpen pencircle scaled .8pt;
    pickup pencircle scaled .8pt;
    draw (5u,0)--(9u,0)--(7u,2u)--cycle;
    pickup pencircle scaled 3pt;
    drawdot (10u,0);
    drawdot (14u,0);
    drawdot (12u,12u);
    endfig;
    \par
    几段直线或曲线可以构成一条路径(path),在路径的末尾加个cycle命令构成封闭路径(closed path)。 \par
    
\subsubsection{预定义形状}
    fullcircle命令是以原点为圆心画的一个单位圆,类似的还有halfcircle,quartercircle,unitsquare等。注意单位正方形的参考点在其左下,使用不同方向的的缩放系数xscaled和yscaled。 \par
\subsubsection{曲线}
    把画直线时坐标点之间--换成..,就得到一条曲线。METAPOST的曲线用三次贝塞尔算法实现,在曲线上使用方向(direction),张力(tension)和曲率(curl)等控制。\par
    \begin{lstlisting}
        filenametemplate "predefined.eps";
        beginfig(3);
        pickup pencircle scaled .8pt;
        draw fullcircle scaled 2u;
        draw halfcircle scaled 2u;
        draw quartercircle scaled 2u shifted (3u,0);
        draw fullcircle xscaled 4u yscaled 2u shifted (9u,0);
        draw unitsquare scaled 2u shifted (12u,-u);
        endfig;
    \end{lstlisting}
\subsection{图形控制}
    \subsubsection{线型,箭头 ,彩色和填充}
    注意:METAPOST不能使用xcolor宏包,只支持rgb和cmyk色彩模式,其自定义颜色的方法如下: \par
    \begin{lstlisting}
        color c[];
        c1 := .9red + .6green + .3blue;
        c2 := (.9,.6,.3);
    \end{lstlisting} \par
    绘图命令一般通过withcolor选项来使用各种颜色。除了为每个绘图命令单独指定颜色,也可以使用全局命令\textbackslash drawoption,使得其后的绘图命令都使用某种颜色。 \par
\subsection{图形变换}
    对路径进行平移(shifted)[参数:移到的坐标点],旋转(rotated)[参数:角度,旋转中心为原点],定点旋转(rotatedaround)[参数:旋转中心],镜像(reflectedabout)[参数:两点确定的一条直线],倾斜(slanted)[倾斜比]等变换操作。 \par

\subsection{标注}
    label在指定位置加文字标注。命令的8个后缀,top,bottom,lft,rt;ulft左上,urt右上,llft左下,lrt右下。dotlabel命令加标注的同时画了个点。\par
    \begin{figure}[htbp]
        \centering
        \includegraphics{9.jpg}
    \end{figure}
    \begin{lstlisting}
        label.rt("文字",(4u,4u))
    \end{lstlisting}

\subsection{编程}
\subsubsection{数据类型和变量}
    如前面所用的缩放系数u是numeric,点的坐标是pair。除缺省的numeric外,其他变量在使用时需要用数据类型来显示声明。相同类型的变量可以在一行语句中声明,但带下标的变量不能放在同一行。 \par

\section{PSTricks}
    PSTricks是基于PostScript的宏包,可以直接在\LaTeX{}文档中插入绘图命令。以下为可以和PSTricks配合使用的辅助宏包:\par
    \begin{figure}[htbp]
        \centering
        \includegraphics{10.jpg}
    \end{figure}
\subsection{Begin}
    缺省单位是cm,绘图命令一般放在pspicture 环境中,参数为矩形的左下角和右上角,从左下角的原点开始可以省略该点坐标。 \par
    \begin{lstlisting}
        \passet{unit=10pt}
        \begin{pspicture}(0,0)(4,2)
            ...
        \end{pspicture}        
    \end{lstlisting} \par
    pst-pdf宏包生成包含PSTricks图形的EPS,然后根据需要转换为PDF。每个pspicture环境中的内容会自成一页,方便插入文档。 \par
\subsection{基本图像对象}
     \subsubsection{点和直线}
    \begin{lstlisting}
    \begin{pspicture}(-.2,-.2)(14,2.2)
       \psdot(0,0) %\dot命令画一个点
       \psdots(4,0)(2,2) %\dots命令画多个点
       \psline(5,0)(7,2)(9,0) 
       \psline命令把多个点用直线段连接起来,线段间缺省为尖角。
       \psline[linearc = .3](10,0)(12,2)(14,0)
    \end{pspicture}
    % \end{lstlisting}

    \begin{pspicture}(-.2,-.2)(14,2.2)
       \psdot(0,0) 
       \psdots(4,0)(2,2)
       \psline(5,0)(7,2)(9,0)
       \psline[linearc = .3](10,0)(12,2)(14,0)
    \end{pspicture}
    \psline[linewidth=2pt,linearc=.25]{->}(4,2)(0,1)(2,0)
    \pscircle[fillcolor=yellow]{1.5}
    
    矩形命令psframe,参数就是矩形左下角和右上角的坐标。多边形用pspolygon命令,语法和psline相似,但其会形成封闭路径。矩形和多边形可以设置圆角。\par
    \begin{lstlisting}
        \begin{pspicture}(19,3)
            \psframe(0,0)(4,3)
            \psframe[framearc=.3](5,0)(9,3)
            \pspolygon(10,0)(14,0)(12,3)
            \pspolygon[linearc=.3](15,0)(19,0)(17,3)
        \end{pspicture}
    \end{lstlisting}
\subsubsection{圆,椭圆,圆弧,扇形}
圆形用 \textbackslash pecircle,参数是圆心和半径。椭圆psellipse,参数:中心,长径,短径。\par

\subsubsection{曲线}
pscurve曲线命令,平滑曲线连接。psecurve命令显示曲线的两个端点;psccurve命令则把曲线封闭起来,showpoints参数用来指示是否显示曲线的构成点。贝塞尔曲线用psbezier命令,参数是曲线的控制点。抛物线用psparabola命令,参数:抛物线通过某点,抛物线的顶点。\par

\subsubsection{网格和坐标轴}
制图用到坐标和网络。psgrid命令输出矩形网格,三个参数点。坐标标注在通过第一点的两条直线上,第二和第三点是矩形的两个对角顶点。当第一个参数省略时,坐标标注在通过第一个顶点的两条矩形边上。坐标轴用pst-plot宏包的 \textbackslash psaxes命令。\par

\section{PGF}
\subsection{准备工作}
一般不直接使用PGF底层命令,而是通过TikZ来调用。在引用tikz宏包之前,需设置PGF系统驱动。比如dvipdfmx的设置方法如下,使用pdflatex和xelatex时,知道驱动是谁。 \par
\begin{lstlisting}
    \def\pgfsysdriver{pgfsys-dvipdfmx.def}
    \usepackage{tikz}
\end{lstlisting} \par
PGF的缺省长度单位是1cm,TikZ提供 \textbackslash tikz命令和tikzpicture环境,具体的绘图指令可以放在 \textbackslash tikz后面,也可以放在tikzpicture环境,两者效果相同。\par
\begin{lstlisting}
    \tikz....%绘图命令
    \begin{tikzpicture}
        .... %绘图命令
    \end{tikzpicture}
\end{lstlisting} \par
为节省编译时间,可以用preview宏包生成独立图形文件,虽然这样做不是必须的。 \par
\begin{lstlisting}
    \documentclass{article}
    \usepackage[active,tightpage,xetex]{preview}
    \usepackage{tikz}

    \begin{document}
        \begin{preview}
            \begin{tikzpicture}
                ...
            \end{tikzpicture}
        \end{preview}
    \end{document}
\end{lstlisting}
\subsection{基本图形对象}
\subsubsection{直线和矩形}
PGF绘图命令语法和METAPOST类似,\textbackslash draw称为一个命令,它后面的--(用来画直线),cycle(用来封闭路径),rectangle等操作,[rounded corners]称为一个选项,用来加圆角。\par
\begin{lstlisting}
    \draw (0,0)--(4,0)--(2,2)--(0,0) %起始点是(0,0);
    \draw (5,0)--(9,0)--(7,2)--cycle %起始点(7,2);
    \draw [rounded corners] (10,0)--(14,0)--(12,2)--cycle;
    \draw (15,0) rectangle (19,2);
    %起始点(15,0),另一对角顶点(19,2);
    \draw [rounded corners] (20,0) rectangle (24,2);
\end{lstlisting}
% \begin{preview}    
% \begin{tikzpicture}
%     \draw (0,0)--(4,0)--(2,2)--(0,0); %起始点是(0,0)
%     \draw (5,0)--(9,0)--(7,2)--cycle; %起始点(7,2)
%     \draw [rounded corners] (10,0)--(14,0)--(12,2)--cycle;
%     \draw (15,0) rectangle (19,2);
%     %起始点(15,0),另一对角顶点(19,2)
%     \draw [rounded corners] (20,0) rectangle (24,2);
% \end{tikzpicture}
% \end{preview}

\subsubsection{圆,椭圆,弧形}

\subsubsection{曲线}

\subsubsection{网格}

\subsection{图形控制}

\subsubsection{箭头}

\subsubsection{线宽和线性}

\subsubsection{颜色和填充}

\subsubsection{渐变和阴影}

\subsubsection{样式}

\subsection{图形变换}


\subsection{示意图}

\subsubsection{节点}
PGF中的节点(node)可以是简单的标签,也可以各形状的边框,还可以有各种复杂的属性。
\tikzset{
    box/.style = {rectangle,rounded corners=5pt,minimum width =50pt,minimum height=20pt,inner sep=5pt,draw=Silver,fill=Lavender}
}

\subsubsection{流程图}
% \node[box] (tex) at(0,0) {.tex};

\subsubsection{树}


\section{表格}
\subsection{简单表格}
tabular环境提供最简单的表格功能,\textbackslash hline命令表示横线,|表示竖线;\& 表示分列。tabular环境语法:[纵向对齐] \{ 横向对齐和分隔符 \} \par
\begin{tabular}{|l|c|r|}
    \hline
    操作系统    &   发行版  &   编辑器 \\
    \hline
    Windows    &    MikTeX  &   TexMakerX \\
    \hline
    Unix/Linux  &   teTeX   &   Kile \\
    \hline
\end{tabular} \par
表格也有类似图像的浮动环境table,可把表格简化为三线表。三个横线一样粗细,也可以使用booktabs宏包,分别用 \textbackslash toprule,midrule,bottomrule等命令表示。 \par
\begin{table}[htbp]
    \centering
    \begin{tabular}{ccc}
        \toprule
        操作系统    &   发行版  &   编辑器 \\
        \midrule
        Windows    &    MikTeX  &   TexMakerX \\
        Unix/Linux  &   teTeX   &   Kile \\
        \bottomrule    
    \end{tabular}
\end{table}

\subsection{表格宽度控制}
需要具体某列的宽度,将其参数从l,c,r改为p{宽度},此时纵向对齐方式为居顶。使用宽度控制参数之后,表格内容缺省居左对齐。可使用列前置命令的语法:> \{ 命令 \} 列参数。列前置命令仅对紧邻其后的一列有效。\par 

\begin{table}[htbp]
    \centering
    \begin{tabular}{p{80pt}p{80pt}p{80pt}}
        \toprule
        操作系统    &   发行版  &   编辑器 \\
        \midrule
        Windows    &    MikTeX  &   TexMakerX \\
        Unix/Linux  &   teTeX   &   Kile \\
        \bottomrule    
    \end{tabular}
\end{table}
控制整个表格的宽度,可以使用tabularx宏包的同名环境,其中X参数表示可以折行(自动换行)。语法:\{ 表格宽度 \} \{ 横向对齐,分隔符,拆行 \} ,如果想把纵向对齐方式改为居中和居底,可以使用array宏包,提供两个对齐参数:m{宽度},b{宽度}。\par
\begin{lstlisting}
    \begin{table}
        \centering
        \begin{tabularx}{350pt}{lXlX} 
            %350pt总宽度,四列对齐方式,
            \toprule
            \midrule
            \bottomrule
        \end{tabularx}
    \end{table}
\end{lstlisting}

\begin{table}[htbp]
    \centering
    \begin{tabular}{p{2cm}p{2cm}p{2cm}p{2cm}}%表示某列可以折行
    \toprule
    设置列宽为p2cm超过宽度会自动换行  &   可以折行的第二列,无边落木萧萧下,无尽长江滚滚来   &    不可折行的第三列    &   可折行的第四列,欲把西湖比西子,人面桃花相映红 \\
    kk & lk & opo &  oo \\
    ff & hh & jjj &  ghh \\
    \bottomrule
    \end{tabular}
\end{table}

\subsection{跨行跨列}
表格单元需要横跨多列,使用 \textbackslash multicolumn命令。同时使用booktabs宏包的 \textbackslash cmidrule 命令横跨几列横线。\par
语法:\textbackslash multicolumn \{ 横跨列数 \} \{ 对齐方式 \} \{ 内容 \} \par
语法:\textbackslash cmidrule \{起始列-结束列\} \par
语法: \textbackslash multirow \{竖跨行数\} \{跨度\} \{内容\} 
跨行的表格可以使用multirow宏包的 \textbackslash multirow命令。
\begin{table}[htbp]
    \centering
    \begin{tabular}{lll}
       \toprule
       & \multicolumn{2}{c}{常用工具} \\
      \cmidrule{2-3} %横线
      操作系统 & 发行版 & 编辑器 \\
      \midrule
      Windows   &   MikTex & TexMakerX \\
      Unix  &   teTex   &   Kile \\
      \bottomrule
    \end{tabular}
\end{table}

\subsection{数字表格}
表格包含大量数字时,手工调整小数点和位数对齐很麻烦,可使用warpcol宏包,为tabular环境提供了列对齐参数P,语法:P \{-m.n\},其中m和n分别是小数点前后的位数,数字前的负号可选。使用multicolumn可以保护表头,防止被P参数误伤。 \par  
\begin{table}[htbp]
    \centering
    \begin{tabular}{P{2.5}P{-2.5}}
        \toprule[1pt]
        \multicolumn{1}{c}{数学常数} &
        \multicolumn{1}{c}{物理常数} \\
        \midrule
        3.14159 & 2.99792 \\
        27.18281 & -17.58819 \\
        \bottomrule
    \end{tabular}
\end{table}

\subsection{长表格}
表格太长跨页时,可以用longtable宏包。具体工作如下:\par
\begin{compactenum}
    \item 用longtable环境取代tabular环境;
    \item 在表格开始部分定义每页首页出现的通用表头,表头最后一行末尾不用 \textbackslash \textbackslash 换行,而是加一个 \textbackslash endhead命令;
    \item 定义首页表头(和通用表头不同时),最后一行的结尾为 \textbackslash endfirsthead命令结尾。
    \item 以 \textbackslash endfoot命令结尾的通用表尾;
    \item 以 \textbackslash endlastfoot命令结尾的末页结尾(如果和通用表尾不同的话);
    \item 表格的具体内容;
\end{compactenum}

\begin{longtable}{ll}
    %定义每页的表头内容,两行内容
    \multicolumn{1}{r}{上一页} \\
    \toprule
    作者 & 作品 \\
    \midrule
    \endhead

\caption{长表格演示} \\

%定义首页的表头
    \toprule
    作者 & 作品 \\
    \midrule
    \endfirsthead

    \bottomrule
    \multicolumn{2}{r}{下一页 \dots} \\
    \endfoot
    %通用表的结尾

    \bottomrule
    \endlastfoot
    %最后的一页如果不同时
    白居易 & 浔阳江头夜送客,枫叶荻花秋瑟瑟。 \\
     & 主人下马客在船,举酒欲饮无管弦。 \\
          & 醉不成欢惨将别,别时茫茫江浸月。 \\
           & 主人下马客在船,举酒欲饮无管弦。 \\
          & 醉不成欢惨将别,别时茫茫江浸月。 \\
           & 主人下马客在船,举酒欲饮无管弦。 \\
          & 醉不成欢惨将别,别时茫茫江浸月。 \\
           & 主人下马客在船,举酒欲饮无管弦。 \\
          & 醉不成欢惨将别,别时茫茫江浸月。 \\
           & 主人下马客在船,举酒欲饮无管弦。 \\
          & 醉不成欢惨将别,别时茫茫江浸月。 \\
           & 主人下马客在船,举酒欲饮无管弦。 \\
          & 醉不成欢惨将别,别时茫茫江浸月。 \\
           & 主人下马客在船,举酒欲饮无管弦。 \\
          & 醉不成欢惨将别,别时茫茫江浸月。 \\
           & 主人下马客在船,举酒欲饮无管弦。 \\
          & 醉不成欢惨将别,别时茫茫江浸月。 \\
           & 主人下马客在船,举酒欲饮无管弦。 \\
          & 醉不成欢惨将别,别时茫茫江浸月。 \\
           & 主人下马客在船,举酒欲饮无管弦。 \\
          & 醉不成欢惨将别,别时茫茫江浸月。 \\
           & 主人下马客在船,举酒欲饮无管弦。 \\
          & 醉不成欢惨将别,别时茫茫江浸月。 \\
           & 主人下马客在船,举酒欲饮无管弦。 \\
          & 醉不成欢惨将别,别时茫茫江浸月。 \\
           & 主人下马客在船,举酒欲饮无管弦。 \\
          & 醉不成欢惨将别,别时茫茫江浸月。 \\
           & 主人下马客在船,举酒欲饮无管弦。 \\
          & 醉不成欢惨将别,别时茫茫江浸月。 \\
           & 主人下马客在船,举酒欲饮无管弦。 \\
          & 醉不成欢惨将别,别时茫茫江浸月。 \\
           & 主人下马客在船,举酒欲饮无管弦。 \\
          & 醉不成欢惨将别,别时茫茫江浸月。 \\
           & 主人下马客在船,举酒欲饮无管弦。 \\
          & 醉不成欢惨将别,别时茫茫江浸月。 \\
           & 主人下马客在船,举酒欲饮无管弦。 \\
          & 醉不成欢惨将别,别时茫茫江浸月。 \\
           & 主人下马客在船,举酒欲饮无管弦。 \\
          & 醉不成欢惨将别,别时茫茫江浸月。 \\
\end{longtable} \par

\subsection{宽表格}
表格太宽时用rotating宏包,用sidewaystable环境替代table环境即可。\par
\begin{sidewaystable}[htbp]
    \caption{主流英文词典}
    \label{tab:dict}
    \centering
    \begin{tabularx}{550pt}{Xllcrrr} %总宽550pt,7行,注意是tabularx,X!!!
        \toprule
        Title   &   Abbr    &   Publisher   &   Year    &   Pages   &   Entries &   Price \\
        \midrule
        Oxford English Dict, 2nd Ed & OED & OxfordUniv & 1989 & 21728 & 616500 & 995 \\
        \midrule
        Shorter Oxford English Dict,7th Ed & SOED & OxfordUniv &2007 & 3888 & 600000 & 175 \\
        \bottomrule
    \end{tabularx}
\end{sidewaystable}

\subsection{彩色表格}
用colortbl宏包,提供 \textbackslash columncolor,rowcolor,cellcolor分别设置列,行,单元格的颜色。columncolor需要放到前置命令里,rowcolor,cellcolor分别放到行,单元格之前。colortbl宏包可以使用xcolor宏包的色彩模型;两者同时,则前者不能直接加载,需要通过后者的选项table来加载。三个命令同时使用,顺序为:单元格,行,列。 \par
\begin{table}[htbp]
    \centering
    \begin{tabular}{l>{\columncolor{Yellow}}ll}
        \rowcolor{Red}
        操作系统    &   发行版  &   编辑器 \\
        Windows     &   MikTex &    TexMakerX \\
        \rowcolor{Green}
        Linux       &   \cellcolor{Lavender}teTex   &   Kile \\
        \rowcolor{Blue}
        通用        &   TeX Live    &   TeXworks \\
    \end{tabular}
\end{table}
xcolor宏包的rowcolors命令可以设置奇偶行的颜色,语法:\{起始行\} \{奇数行颜色\} \{偶数行颜色\}。hiderowcolors命令用来暂停显示前面设置的奇偶行颜色,否则后续其他表格会继续显示颜色,另一个命令showrowcolors可以重新激活奇偶行颜色设置。 \par

% \begin{table}[htbp]
%     \centering
%     \rowcolors{1}{White}{Blue}
%     \begin{tabular}
%         \hline
%         操作系统    &   发行版  &   编辑器 \\
%         Windows     &   MikTex &    TexMakerX \\
%         Linux       &   teTex  &    Kile \\
%         Mac OS      &   MacTeX &    TeXShop \\
%         通用        &   Tex Live &  TeXworks \\
%         \hline
%         \hiderowcolors %用来暂时停止显示前面设置的奇偶行颜色,否则后面其他表格会继续显示颜色。
%     \end{tabular}
% \end{table}


\section{文档的结构}
正文的层次结构,前目录,后索引。长文档通常分为多个文件,与HTML文件类似,PDF也提供超链接功能,通常版权页为偶数页,目录首页为奇数页等。可以在主控文档中引用子文档,\textbackslash include命令会新起一页,不想用新页可以改用 \textbackslash input命令。 \par
\begin{lstlisting}
    %master.tex
    \begin{document}
        \include{chapter1.tex}
        \include{chapter2.tex}
    \end{document}
\end{lstlisting} \par
拆分长文档,当文档很长时,编译一遍需要很长的时间 ,可以使用syntonly宏包,这样编译时只检查语法,而不是生成结果文件。
\begin{lstlisting}
    \usepackage{syntonly}
    ...
    \syntaxonly
\end{lstlisting}

\subsection{标题}
标准文档没有为作者所属单位定义的专门命令。只有title,author,date命令。设置完上述内容后,notitlepage和titlepage用来控制标题是否单独占一页。report和book文档类的标题缺省独占一页,article文档类的标题缺省和正文共占一页。

\subsection{目录}
\textbackslash tableofcontents命令用来生成整个文档的章节目录,可以用 \textbackslash setcounter命令来指定目录的层次深度。不想显示某个章节目录的,可以用*来声明章节。同理,用 \textbackslash listoffigures和 \textbackslash listoftables 命令来生成图目录和表目录。注意,当结构层次变化时,需要编译两次获得新的目录。 \par
第一次编译生成一些中间文件,后缀分别为.toc(目录),.lof(图目录),.lot(表目录);第二次编译则把中间文件和其他内容整合起来。利用宏包caption自定义类似于插图和表格的浮动环境。\par
\textbf{自定义浮动环境}->语法:\textbackslash DeclareCaptionType[选项][环境][名称][目录名]
\begin{lstlisting}
    \DelcareCaptionType[fileext=loe]{example}[例][例目录]
    %指定的loe为例目录中间文件后缀,example为环境名
    %例为浮动环境的标题前缀,例目录为目录的标题
    \begin{example}[h]
        ...
    \end{example}
\end{lstlisting}

\subsection{参考文献}
\subsubsection{thebibliography}
\LaTeX{}最原始的方法是用thebibliography环境和 \textbackslash  bibtem命令来定义参考文献条目。thebibliography环境一般放在文档的末尾,定义好参考文献后,可以用 \textbackslash cite命令在正文中引用条目。
\begin{lstlisting}
    \begin{thebibliography}
    \bibitem{Rowling_1997}{9}
    %参数9表示参考文献条目编号的宽度
    Jock K. Rowling,
    \emph{Harry Potter}.
    Bloomsbury,London,
    1997.
\end{thebibliography}
\end{lstlisting}

% \begin{thebibliography}
%     \bibitem{Rowling_1997}{9}
%     %参数9表示参考文献条目编号的宽度
%     Jock K. Rowling,
%     \emph{Harry Potter}.
%     Bloomsbury,London,
%     1997.
% \end{thebibliography}


\begin{lstlisting}
    %引用定义好的文献
    \cite{Rowling_1997}
\end{lstlisting}

\subsubsection{BibTeX}
thebibliography需要用户自己调整显示格式。另一条路:用数据库文件.bib记录参考文献目录,用样式文件.bst设置显示格式。一般只需维护数据库即可。\textbf{秉承\LaTeX{}内容和格式分离的思想},同理在SGML/DSSSL,HTML/CSS,XML/XSL等技术思路相同。\par
\BibTeX{}将参考文献分为十几种类型,每种类型的参考文献有必选项和可选选项。具体如下: \par
\textbf{manual}手册 
\begin{itemize}
    \item 必选项:title
    \item 可选项:author,organization,address,edition,month,year,note
\end{itemize}
编辑.bib文件时可以用普通文本编辑器,也可以用专门的文献管理软件提高效率,如JabRef。一些其他的文献管理软件或网络服务可以输出为.bib格式,如Endnote,Zotero等。\par
\begin{lstlisting}
    @book{Rowling_1997,
    author  =   "Joanne K.Rowling",
    title   =   "Harry Potter and the Sorcerer",
    publisher = "Bloomsbury,London",
    year    =   "1997"
    }
    %其中每行是一个数据项,第一个数据项是关键字,供引用时用;
    其他的数据项以 名称 = 值的形式出现,值要写在双引号内;数据项间用逗号隔开。
\end{lstlisting}
有了数据后,\LaTeX{}发行版提供4种标准的样式,具体如下:
\begin{compactenum}
    \item \textbf{plain}:参考文献列表按作者姓氏排序,序号为阿拉伯数字。
    \item \textbf{unsrt}:参考文献列表按正文引用顺序排序,序号阿拉伯数字。
    \item \textbf{aloha}:参考文献列表按照作者姓氏排序,序号为作者姓氏加年份。
    \item \textbf{abbrv}:类似plain样式,作者名字,月份,期刊等用缩写。
\end{compactenum}
选定样式后,需要在文档中用 \textbackslash bibliographystyle命令设置样式,然后用 \textbackslash bibliography命令用来输出参考文献列表。前文中的交叉引用的文档需要编译两遍,含参考文献的文档则需要执行4次编译操作。
\begin{lstlisting}
    \bibliographystyle{plain}
    \bibliography{myref}
\end{lstlisting}
\begin{compactenum}
    \item 第一遍xelatex把参考文献条目的关键字写到中间文件.aux中去。
    \item bibtex根据.aux,.bib,.bst生成一个.bbl文件,即参考文献列表。它的内容就是thebibliography环境和一些\textbackslash bibtem命令。
    \item 第二遍xelatex把交叉引用写到.aux中去。
    \item 第三遍xelatex则在正文中正确地显示引用。
\end{compactenum}
\begin{figure}[htbp]
    \centering
    \includegraphics{13.jpg}
\end{figure}
有多个子文档时,可以在每个子文档中用 \textbackslash bibliography命令设置不同的样式。建议使用统一的样式,用xelatex编译主控文档,用bibtex编译各个子文档。\par

\subsubsection{Natbib}
参考文献在正文中的引用通常有两种模式:作者-年份和数字。\LaTeX{}提供的 \textbackslash cite命令只支持数字模式,而natbib宏包则支持两种模式。\par
natbib提供了3种列表样式:plainnat,abbrvnat,unsrtnat,它们的文献参考列表和相应的 \LaTeX{} 标准样式plain,abbrv,unsrt效果相同,只是在引用时可以自由选择作者-年份或数字模式,三种列表样式都有自己的缺省样式,如果定制引用样式,可以使用 \textbackslash setcitestyle命令。 \par

\begin{table}[htbp]
    \centering
    \begin{tabular}{ll}
        \toprule
       引用模式     &   authoryear,number,super \\
       括号         &   round,square,open=char,close=char \\
       引用条目分隔符 &     分号,逗号,citesep=char \\
       作者年份分隔符 &     aysep=char \\
       共同作者年份分隔符 & yysep=char \\
       注解分隔符   &   notesep=texy \\
        \bottomrule
    \end{tabular}    
\end{table}

natbib提供多种引用命令,最基本的是 \textbackslash citet和 \textbackslash citep,一般不使用 \LaTeX{}本身提供的 \textbackslash cite命令,其在作者-年份模式下和 \textbackslash citet效果相同,在数字模式下和 \textbackslash citep相同。

\subsection{索引}
makeidx宏包提供索引命令,\textbackslash makeindex命令;注意索引关键字在全文中保持唯一性,一般在文章末尾打印索引。
\begin{lstlisting}
    \usepackage{makeidx}
    \makeindex
    ...
    \begin{document}
        \index{索引关键字}
    \end{document}
\end{lstlisting}

当编译含索引的文档时,用户需要执行3次编译操作。
\begin{compactenum}
    \item 第1遍xelatex把索引条目写到一个.idx文件中去。
    \item makeindex把.idx排序写到一个.ind文件中去。
    \item 第2遍xelatex在 \textbackslash printindex命令的地方引用.ind的内容,生成正确的内容。
\end{compactenum}

\begin{figure}[htbp]
    \centering
    \includegraphics{14.jpg}
\end{figure}

\subsection{超链接}
hyperref宏包提供超链接功能,给文档的内部交叉引用和参考文献加上超链接,\textbackslash hyperref命令对已经定义的标签进行简单包装,\textbackslash url和 \textbackslash href也可以定义外部链接。\par

\begin{lstlisting}
    \label{sec:hyperlink}
    ..
    编号形式的链接:\ref{sec:hyperlink} \\
    文字形式链接:\hyperref\sigma [sec:hyperlink]{链接}
\end{lstlisting}

\begin{lstlisting}
    \url{http://www.dralpha.com/} \\
    \href{https://baidu.com/}{百度} 
\end{lstlisting}

\begin{figure}[htbp]
    \centering
    \includegraphics{11.jpg}
    \includegraphics{12.jpg}
\end{figure}

\subsection{结构名}
每个文档都有自己的名字,一般用来在标题或引用时显示。比如:主目录(Contents),图目录(List of Figures),表目录(List of Tables);章(chapter),节(section),小节(subsection);图(figure),表(Table)。
\begin{lstlisting}
    \renewcommand{\contentsname}{目录}
    \renewcommand{\listfigurename}{图目录}
\end{lstlisting}


\section{布局}
在\LaTeX{}排版对象都是一个盒子,盒子间的相互嵌套到更大的盒子中,排版时页面是最大的盒子,纸张规格有两大标准:公制和美制。
\subsection{历史}
1786年,德国科学家就发现 $\frac{1}{\sqrt{2}}$这个比例可用于分割纸张,在较长的方向一分为二,得到的两张纸也是同样比例。德国标准化学会(DIN)在1922年发布476纸张标准,从面积1 $m^2$的A0(841mm×1189mm)开始,每次减半长边。1961年ISO将A和B系列采纳为推荐标准,1975年变为ISO216标准,其中的B系列比DIN476略大,从1000mm×1414mm的B0开始,1985年发布的ISO269加上C系列,尺寸是A系列和B系列纸张尺寸的几何平均。\par
A系列常用于公文;B系列常用于海报和护照(B7,88mm×125mm);C系列常用于信封,恰好比A系列大一点,如A4纸可以装在C4信封里,对折一下可以装进C5信封,再对折进C6中。\par
大多数国家采用ISO标准,美国和其他几个国家使用比A4宽一点的Letter(8.5in×11in),Legal(8.5in×14in)主要用于法律文件,比普通文件长一大截。1996年美国推出ANSI Y14.1定义了A,B,C,D,E规格。A即Letter,B比A的面积大一倍,C比B大一倍,以此类推。它们的长宽比不一致,B和C比其他三种窄的多,它们的尺寸倒是和A4-A0差不多。\par

\subsubsection{尺寸详解}
\begin{figure}[htbp]
    \centering
    \includegraphics{15.jpg}
\end{figure}
图为一张A4版,尺寸210mm×297mm(597pt×845pt),body为正文区域,Header是页眉,Footer为页脚。图片尺寸介绍如下:
\begin{compactenum}
    \item 页边距,1in
    \item \textbackslash oddsidemargin或evensidemargin,奇数或偶数页左边距,46pt
    \item \textbackslash textwidth,正文宽度360pt,约32汉字
    \item 597pt减去左边的 1in+46pt和中间的360pt,还剩119pt,左右相差1pt。如果双面打印的话,两面的正文部分恰好是重叠
    \item 页边距,1in
    \item \textbackslash topmargin,上边距,18pt
    \item \textbackslash headheight,页眉高度,12pt
    \item \textbackslash headsep,页眉与正文间距,25pt
    \item \textbackslash textheight,正文高度595pt,可以放下38行文字
    \item \textbackslash footskip,正文与页脚基线间距,30pt。比页眉的12pt+25pt小了7pt
    \item \textbackslash 845pt减去上面的全部尺寸,还剩93pt,比上面的1in+18pt多3pt。
\end{compactenum} \par
当字号发生变化时,上面的某些尺寸也发生一定的变化,当把oneside改为twoside,奇偶页的左边距就变成22pt和70pt。但是奇数页右边空白恰好和偶数页左边空白相等,不会给双面打印造成困扰。一般情况无需改动 \LaTeX{}的页面布局缺省设置,有特殊需要时可用 \textbackslash setlength,\textbackslash addtolength设置宏变量的值。geometry宏包提供更高级的用户接口,如下面设置页面尺寸和边距:\par
\begin{lstlisting}
    \usepackage[paperwidth=100mm,paperheight=150mm,margin=20mm]{geometry}
\end{lstlisting}
\begin{lstlisting}
    %也可以单独设置每个边距:
    \usepackage[top=2in,bottom=1in,left=1in,right=1in]{geometry}
    %把页面横过来:
    \usepackage[landscape]{geometry}
\end{lstlisting}

\subsection{页面样式}
页面样式也就是页眉和页脚的内容,有4个样式:\par
\begin{table}[htbp]
    \centering
    \begin{tabular}{ll}
        \toprule
       empty     &   页眉,页脚空白 \\
       plain     &   页眉空白,页脚含居中页码,非book类文档的缺省值\\
       heading   &   页脚空白,页眉含章节名和页码,book类文档缺省值\\
       myheading &   页脚空白,页眉含页码和用户自定义信息\\
        \bottomrule
    \end{tabular}    
\end{table} 
同时可以用 \textbackslash pagestyle和 \textbackslash thispagestyle命令设置整个文档或某页的样式。\par 

\begin{lstlisting}
    \pagestyle{plain} %全局设置
    \thispagestyle{empty} %单页设置

    %自定义样式
    %使用了@特殊符号,第一行用\makeatletter命令声明
    %最后一行用相应的\makeatother命令恢复现场
    \mskeatletter
    \newcommand{\ps@permanentdamagedhead}{
        %自定义奇偶页的页眉和页脚
        \newcommand{\@oddhead}{奇页眉\hfill 右半边}
        %\hill为弹性填充命令
        \renewcommand{\@oddfoot}{\hfill\thepage\hfill}
        \newcommand{\@evenhead}{偶页眉\hfill 奇半边}
        \newcommand{\@evenfoot}{\@oddfoot}
    }
    \makeatother
\end{lstlisting}
自定义页面样式时,还可以使用一些宏变量来显示页码,章节号码和名称等,具体如下表所示:
\begin{table}[htbp]
    \centering
    \caption{页眉页脚常用宏变量}
    \begin{tabular}{lp{10cm}}
        \toprule
       \textbackslash thepage     &   页码 \\
       \textbackslash thechapter  &   章编号\\
       \textbackslash thesection  &   节编号\\
       \textbackslash chaptername &   章起始单词名,Chpter\\
       \textbackslash sectionname &   节起始单词名,Section \\
       \textbackslash leftmark    &   左标记,article文档类中包含section信息,report和book包含chapter信息 \\
       \textbackslash rightmark   &    右标记,在article包含subsection信息,在report和book中包含section信息 \\
        \bottomrule
    \end{tabular}    
\end{table} 
前5个变量可直接定义,左右标记需要用以下命令间接定义:\par
\begin{lstlisting}
    \makeboth{左标记}{右标记} %定义2个标记
    \markboth{右标记} %定义右标记
    %引用时用\leftmark和\rightmark,定义时用\markboth和\markright

    %book类文档中myheadings样式定义
    \def\pa@headings{
        \let\@oddfoot\@empty \let \@evenfoot\@empty
        %清空页脚
        \def\@evenhead{\thepage\hfill\slshape\leftmark}
        %定义偶数页页眉(左页页眉),页码居左,左标记居右
        \def\@oddhead{{\slshape\rightmark}\hfill\thepage}
        %奇数页页眉(右页页眉),右标记居左,页码居右。
    }
\end{lstlisting}\par
fancyhdr宏包提供更灵活的控制和高级语法,如下所示:\par
\begin{lstlisting}
    %怎样为myheadings样式定制左右标记
    \documentclass{book}
    \markboth{左}{右}
    \pagestyle{myheadings}
    ...
    \begin{document}
        左页
        \newpage
        右页
        \newpage
    \end{document}
\end{lstlisting}

\subsection{分栏}
\begin{lstlisting}
    \documentclass[twocolumn]{article}
\end{lstlisting}
\par
multicol宏包提供多达10个栏位,数目可任意切换,各栏的长度相同。注意在multicols环境中对浮动体的支持有限,只能使用带*的版本。且浮动体跨栏位,h选项也会失效,最12早也只出现在下一页的页首,因此建议不让浮动体出现在栏位内。\par
\begin{lstlisting}
    \usepackage{multicol}
    \setlength{\columnsep}{12pt}
    %将栏位之间的距离设为12pt(缺省为10pt)
    \setlength{\columnseprule}{1pt}
    %将栏位间的分割线宽度设为1pt(缺省不显示)
    \begin{multicols}{2}
        ...
    \end{multicols}
\end{lstlisting}

\subsection{分页}
\TeX{}一般自动分页,针对浮动体较多的情况,自动分页效果可能不是我们想要的,可以手工插入分页命令。\par
\begin{lstlisting}
    \newpage
    \pagebreak[3]
    %参考数值1-4,4表示强烈分页
    \nopagebreak[2]
    %强烈程度1-4,4表示强烈反对
    \clearpage
    %浮动体较多,此命令要求\TeX{}排完此前所有的浮动体
\end{lstlisting}

%连接文件显示的编写
\lstinputlisting[
    style       =   Python,
    captionpos  =   {\bf demo.py},
    label       =   {demo.py}
]{demo.py}
%表示代码文件的相对位置

\section{符号}
\subsection{\TeX 标志符号}
%基本符号
\TeX{} ,\LaTeX{}, \LaTeXe{}

%需要导入xltxtra宏包
\XeLaTeX{}

%texnames宏包提供
\AmSTeX{},\AmS-\LaTeX{}

\BibTeX{},\LuaTeX{}

%mflogo宏包提供
\METAFONT{},\MF{},\MP{}

\begin{lstlisting}
%基本符号标志
\TeX{} ,\LaTeX{}, \LaTeXe{}
%需要导入xltxtra宏包
\XeLaTeX{}
%texnames宏包提供
\BibTeX{},\LuaTeX{}
\AmSTeX{},\AmS-\LaTeX{}
%mflogo宏包提供
\METAFONT{},\MF{},\MP{}
\end{lstlisting}
%符号说明:
% \\ 纯换行 ; \par 换行带缩进
%\subsection子标题
%输出 \[反斜线] \textbackslash
%输出 % \%
\section{简历的制作}



\end{document}
